%Example of use of oxmathproblems latex class for problem sheets
\documentclass{oxmathproblems}

%(un)comment this line to enable/disable output of any solutions in the file
%\printanswers

%define the page header/title info
\oxfordterm{Macroeconomics II}
\course{Práctica 3 - Cesar Ramos}
\sheetnumber{1}
\sheettitle{Crecimiento Económico} %can leave out if no title per sheet

% add further contact details to footer if desired,
%e.g. email address, or name and email address
\contact{Universidad Mayor de San Andrés - Carrera de Economía}

%--------------------------------------------------------------------%
%--------------------------------------------------------------------%
%--------------------------------------------------------------------%
\usepackage{tikz}
\begin{document}
\begin{questions}
%--------------------------------------------------------------------%
%--------------------------------------------------------------------%
%--------------------------------------------------------------------%
\miquestion
\begin{parts}
  \part 
  Considere el modelo de Harrod y Domar con las siguientes ecuaciones:
  \[{S_t} = {I_t}\]
  \[{I_t} = {K_{t + 1}} - {K_t} + \delta {K_t}\]
  \[\theta  = \frac{{{K_t}}}{{{Y_t}}}\]
  \begin{subparts}
     \subpart
    Encuentre la tasa de crecimiento del capital y la tasa de crecimiento garantizada ¿Qué variable determina el crecimiento según este modelo?
    \paragraph{Resp.}
    El ahorro determina el crecimiento de esta economía es decir la tasa de ahorro que es exógena, llevamos todo a términos per cápita: $$s_t=i_t$$ también $$i_t=\Delta K_t / N_t + \delta k_t$$  
    El aporte del capital a la producción:
    $$PMeK = \theta = K_t / Y_t$$
    Donde:
    $$S_t = s*Y_t$$
    Inversión Bruta
    $$I=\Delta K + \delta K$$
    Inversión Neta
    $$\Delta K = I - \delta K$$
    Reemplazamos la identidad ahorro - inversión en:
    $$\Delta K = s*Y - \delta K$$
    Dividimos entre K la ecuación
    $$\dfrac{\Delta K}{K}  = s*\dfrac{Y}{K} - \delta $$
    La relación capital producto es $\theta = \dfrac{Y}{K}$
    $$\dfrac{\Delta K}{K}  = \dfrac{s}{\theta} - \delta $$
  \subpart
  Levante el supuesto de la inexistencia de una tasa de depreciación. Encuentre la tasa de crecimiento del capital y la tasa de crecimiento garantizada.
  $$\delta = 0$$
  $$g_K = \dfrac{\Delta K}{K}  = \dfrac{s}{\theta} $$
  \subpart
  Introduzca a la población. Encuentre la tasa de crecimiento del capital per – cápita y la tasa de crecimiento garantizada per – cápita.
  Donde $n$ es la tasa de crecimiento poblacional  $$\frac{{\dot L}}{L} = n$$:
  $$g_K = \dfrac{\Delta K}{K}  = \dfrac{s}{\theta} - n$$
    \end{subparts}
\end{parts}

\begin{solution}
                                     
\end{solution}
%--------------------------------------------------------------------%
%--------------------------------------------------------------------%
%--------------------------------------------------------------------%
\miquestion
\begin{parts}
  \part Considere las siguientes ecuaciones para el modelo de Solow:
  \[{Y_t} = AK_t^\alpha L_t^{1 - \alpha }\]
  \[{I_t} = {{\dot K}_t} + \delta {K_t}\]  
  \begin{subparts}
  \subpart
    Verifique que esta función de producción cumple con las propiedades de una función de producción neoclásica.
    %_______________________________________________
    \\
    \textbf{Productividades marginales de los factores positivos:}
    
    \[\frac{{\partial Y}}{{\partial K}} = \alpha A{K^{\alpha  - 1}}{L^{1 - \alpha }} \to \frac{{\partial Y}}{{\partial K}} = \alpha A{k^{\alpha  - 1}} > 0\]
    
     \[\frac{{\partial Y}}{{\partial L}} = (1 - \alpha )A{K^\alpha }{L^{ - \alpha }} \to \frac{{\partial Y}}{{\partial L}} = (1 - \alpha )A{k^\alpha } > 0\]
    %______________________________________________
    \textbf{Productividades marginales de los factores decrecientes:}
    
    \[\frac{{{\partial ^2}Y}}{{\partial {K^2}}} =  - \alpha *(1 - \alpha )A{K^{\alpha  - 2}}{L^{1 - \alpha }} < 0 \to \frac{{{\partial ^2}Y}}{{\partial {K^2}}} =  - \alpha *(1 - \alpha )\frac{Y}{{{K^2}}} < 0\]
    
    \[\frac{{{\partial ^2}Y}}{{\partial {L^2}}} =  - \alpha *(1 - \alpha )A{K^\alpha }{L^{ - \alpha  - 1}} < 0 \to \frac{{{\partial ^2}Y}}{{\partial {L^2}}} =  - \alpha *(1 - \alpha )\frac{Y}{{{L^2}}} < 0\]
    
    %______________________________________________
    \textbf{Condiciones de Inada:}
   
   \[\mathop {\lim }\limits_{k \to 0} ( - \alpha *(1 - \alpha )\frac{Y}{{{K^2}}}) =  - \infty \]
   
   \[\mathop {\lim }\limits_{k \to \infty } ( - \alpha *(1 - \alpha )\frac{Y}{{{K^2}}}) = 0\]
   
  \subpart  
    Encuentre la Elasticidad Producto – Capital per cápita.
    \textbf{Resp.} La eslaticidad producto - capital per cápita:
    $$\dfrac{Y_t}{L_t} = y_t = A*K_t^\alpha L_t^{1-\alpha} // \dfrac{1}{L_t}$$
    
    $$y_t = A*\dfrac{K_t^\alpha}{L_t^{\alpha}}$$
    
    $$y_t = A*k_t^\alpha$$
    
    $$\eta_{y_t,k_t} = A* \alpha * k_t^{\alpha -1} * \dfrac{k_t}{y_t}$$
    
    $$\eta_{y_t,k_t} =  \alpha * \dfrac{A*k_t^{\alpha}}{A*k_t^{\alpha}}$$
    
    $$\eta_{y_t,k_t} =  \alpha$$
    
\subpart    
    Encuentre la función de producción per cápita, acumulación de capital per cápita, la tasa de crecimiento del capital per Cápita, la tasa de crecimiento del producto per cápita (Convergencia absoluta). Realice gráficos.
    
    \textbf{Producción Per cápita}
    
     $$\dfrac{Y_t}{L_t} = y_t = A*K_t^\alpha L_t^{1-\alpha} // \dfrac{1}{L_t}$$
    
    $$y_t = A*\dfrac{K_t^\alpha}{L_t^{\alpha}}$$
    
    $$y_t = A*k_t^\alpha$$
    
    \textbf{Ley de Acumulación del capital}
    
    \[{I_t} = {{\dot K}_t} + \delta {K_t}\] 
    $${I_t} = \Delta K_t + \delta {K_t}$$
    $$\Delta K_t = {I_t} - \delta {K_t} // \dfrac{1}{L_t}$$ 
    $$\dfrac{\Delta K_t}{L_t} = i_t - \delta {k_t}$$
    $$\dfrac{\Delta K_t}{K_t}*\dfrac{K_t}{L_t} = i_t - \delta {k_t}$$
    Operamos el término siugiente:
   
    \[\begin{array}{l}
\mathop {\left( {\frac{K}{{AL}}} \right)}\limits^ \bullet   = \frac{{\mathop K\limits^ \bullet  AL - K(\mathop {AL}\limits^ \bullet  )}}{{{{(AL)}^2}}} = \frac{{\mathop K\limits^ \bullet  AL}}{{{{(AL)}^2}}} - \frac{{K(\mathop {AL}\limits^ \bullet  )}}{{{{(AL)}^2}}} = \frac{{\mathop K\limits^ \bullet  }}{{AL}} - \frac{{K\mathop A\limits^ \bullet  L + KA\mathop L\limits^ \bullet  }}{{{{(AL)}^2}}}\\
\mathop k\limits^ \bullet   = \frac{{\mathop K\limits^ \bullet  }}{{AL}} - \frac{{K\mathop A\limits^ \bullet  L}}{{{{(AL)}^2}}} + \frac{{KA\mathop L\limits^ \bullet  }}{{{{(AL)}^2}}}\\
\mathop k\limits^ \bullet   = \frac{{\mathop K\limits^ \bullet  }}{{AL}} - \frac{{\mathop A\limits^ \bullet  }}{A}\frac{K}{{AL}} + \frac{{\mathop L\limits^ \bullet  }}{L}\frac{K}{{AL}}\\
\mathop k\limits^ \bullet   = \frac{{\mathop K\limits^ \bullet  }}{{AL}} - a*k + n*k\\
\frac{{\mathop K\limits^ \bullet  }}{{AL}} = \mathop k\limits^ \bullet   + a*k - n*k
\end{array}\]
     
     $$\mathop k\limits^ \bullet   = s*f(k) - (n + a + \delta )*k$$
        
    \medskip
    
    \newpage
    
    \begin{center}
        \textbf{Gráfico. Capital per cápita, depreciación y ahorro}
    \end{center}
    %_________________________________________________
    %*************************************************
    \begin{tikzpicture}
    %Axis
    \draw[thick,<->] (0,10) node[above]{Depreciación, Ahorro}--(0,0)--(10,0) node[right]{$k$}; 
    \node [below left] at (0,0) [$0$]; 
    %Line 
    \draw(0,0)--(9,9) node[right]{$\delta k$}; 
    %Curve 
    \draw(0,0) ..controls (1,5) and (5,6) .. (10,7) node[right]{$sf(k)$};
    \end{tikzpicture}
    %*************************************************
    %_________________________________________________

  \end{subparts}
\end{parts}


\begin{solution}
  
\end{solution}


%--------------------------------------------------------------------%
%--------------------------------------------------------------------%
%--------------------------------------------------------------------%
\miquestion
\begin{parts}
  \part El modelo de Solow viene dado por las siguientes ecuaciones:
   \[{Y_t} = AK_t^\alpha L_t^{1 - \alpha }\]
  \[{I_t} = {{\dot K}_t} + \delta {K_t}\] 
  \begin{subparts}
    \subpart Si el progreso tecnológico es 2, la Elasticidad Producto – Capital per – cápita es igual a 0,4; la tasa de depreciación 0,08; la tasa de ahorro 0,25 y la tasa de crecimiento de la población 0,04. Encontrar el nivel de capital per – cápita de estado estacionario, la producción per – cápita. Grafique los resultados.
    
     $$y_t = A*k_t^\alpha$$
     
     $$\mathop k\limits^ \bullet   = s*f(k) - (n + \delta )*k$$
     
     Reemplazando los datos tenemos:
     
     $$y_t = 2*k_t^{0.4}$$
     
     $$\mathop k \limits ^ \bullet   = 0.25*y_t - (0.04 + 0.8 )*k$$
    
    $$\mathop k \limits ^ \bullet   = 0.25*2*k_t^{0.4} - (0.04 + 0.8 )*k$$
    
    $$\mathop k \limits ^ \bullet   = 0.5*k_t^{0.4} - 0.84*k$$
    
    \subpart 
    Encuentre el capital per – cápita de oro, compare el resultado con el obtenido en el inciso i), apóyese con un gráfico.
    $$c^*=f(k^*)+(\delta + n)*k^*$$
    $$\dfrac{dc^*}{dk^*}=f'(k^*)-(\delta +n)=0$$
    $$k^{oro} = A\alpha k^{\alpha -1} = (\delta +n)$$
    $$k^{oro} = (\dfrac{(\delta +n)}{A * \alpha})^{\dfrac{1}{\alpha - 1}}$$
    \subpart 
    Si la tasa de ahorro aumenta hasta 0,3 ¿Qué sucede con el nivel de capital per – cápita de estado estacionario, la producción per – cápita, la tasa de crecimiento del capital per – cápita, la tasa de crecimiento del producto per – cápita? Grafique y discuta los resultados.
    
     $$\mathop k\limits^ \bullet   = s*f(k) - (n + \delta )*k // 1/k_t$$
     
     $$\dfrac{\mathop k\limits^ \bullet}{k} = s*k_t^{-0.6}-(n+\delta)=0$$
     En estado estacionario la tasa de crecimiento del stock de capital es 0.
     
     $$\gamma _{\mathop k\limits^ \bullet}=0$$
    
    $$k_t^{-0.6}=\dfrac{(n+\delta)}{s}$$
    
    Reemplazando los datos:
    $$k_t^{-0.6}=\dfrac{(0.08+0.04)}{0.3}$$
    
    $$k_t=0.04^{- \dfrac{1}{0.6}}$$
    
    \subpart 
    Si la tasa de depreciación cae hasta 0,06 ¿Qué sucede con el nivel de capital per – cápita de estado estacionario, la producción per – cápita, la tasa de crecimiento del capital per – cápita y la tasa de crecimiento del producto per – cápita? Grafique los resultados
    
    $$k_t^{-0.6}=\dfrac{(0.04+0.06)}{s}$$
    
    $$k_t^{-0.6}=\dfrac{(0.1)}{0.4}$$
    
    $$k_t=0.25^{- \dfrac{1}{6}}$$
    
    
    %\subpart 
    %¿Qué valor debería tomar la tasa de ahorro para que el nivel de capital per – cápita sea igual capital per – cápita de oro?
    
    %$$k^{oro} = (\dfrac{(\delta +n)}{A * \alpha})^{\dfrac{1}{\alpha - 1}}$$
    
    
  \end{subparts}
\end{parts}

\begin{solution}
  The solution would go here
\end{solution}

%--------------------------------------------------------------------%
%--------------------------------------------------------------------%
%--------------------------------------------------------------------%
\miquestion
\begin{parts}
  \part El modelo Solow – Swan se caracteriza por las siguientes ecuaciones:
    \[{Y_t} = AK_t^\alpha L_t^{1 - \alpha }\]
    \[ {{\dot K}_t} = {I_t} - \delta {K_t}\] 
  \begin{subparts}
    \subpart 
    a)  El progreso tecnológico es 2, la Elasticidad Producto – Capital per – cápita es igual a 0,55; la tasa de depreciación 0,9; la tasa de ahorro 0,4 y la tasa de crecimiento demográfica es 0,05. Encontrar el nivel de capital per – cápita de estado estacionario, la producción per – cápita. Grafique los resultados. 
    
    \textbf{Producción per cápita}
    $$y_t = k_t^{\alpha}$$
    
    \textbf{Ley de moviemiento del capital}
     $$\mathop k\limits^ \bullet   = s*f(k) - (n + \delta )*k $$
     
     \textbf{Tasa de crecimiento del Stock de capital}
     
     $$\dfrac{\mathop k\limits^ \bullet}{k} = s*k_t^{\alpha - 1}-(n+\delta)=0$$
    
    $$s*k_t^{\alpha - 1}=(n+\delta)$$
    
    $$k_t^{\alpha - 1}=\dfrac{n+\delta}{s}$$
    
    $$k_t^*=(\dfrac{n+\delta}{s})^{\dfrac{1}{\alpha - 1}}$$
    
    \subpart 
    Encuentre el capital per – cápita de oro y el nivel de producción per – cápita de oro, ¿El capital de estado estacionario es mayor o menor al capital de oro (per – cápita)? compare el resultado con el obtenido en el inciso a), respalde su resultado con un gráfico.
    
    $$k_t^*=(\dfrac{0.05+0.9}{0.4})^{\dfrac{1}{0.55 - 1}}$$
    
    $$k_t=(\dfrac{0.95}{0.4})^{\dfrac{1}{-0.45 }}$$
    
    $$k_t=0,14$$

    \subpart Si la tasa de ahorro aumenta hasta 0,5 ¿Qué sucede con el nivel de capital per – cápita de estado estacionario, la producción per – cápita, la tasa de crecimiento del capital per – cápita, la tasa de crecimiento del producto per – cápita? Grafique y discuta los resultados.
    
    $$k_t=(\dfrac{0.95}{0.4})^{\dfrac{1}{-0.45 }}$$
    
    $$k_t=0,24$$
    %_________________________________________________
    %*************************************************
    \begin{tikzpicture}
    %Axis
    \draw[thick,<->] (0,10) node[above]{Depreciación, Ahorro}--(0,0)--(10,0) node[right]{$k$}; 
    \node [below left] at (0,0) [$0$]; 
    %Line 
    \draw(0,0)--(9,9) node[right]{$\delta k$}; 
    \draw(5,0)--(5,5) node[right]{$A$};
    \draw(6.1,0)--(6.1,6.1) node[right]{$B$};
    %Curve 
    \draw(0,0) ..controls (1,5) and (5,6) .. (10,7) node[right]{$s'f(k)$};
    \draw(0,0) ..controls (1,5) and (5,5) .. (10,6) node[right]{$sf(k)$};
    \end{tikzpicture}
    %*************************************************
    %_________________________________________________
    
    \subpart Si la tasa de crecimiento demográfica aumenta hasta 0,06 ¿Qué sucede con el nivel de capital per – cápita de estado estacionario, la producción per – cápita y la tasa de crecimiento del capital per – cápita? Grafique los resultados.
    
    $$k_t^*=(\dfrac{0.06+0.9}{0.4})^{\dfrac{1}{0.55 - 1}}$$
    
    $$k_t^*=0.1429$$
    
    %_________________________________________________
    %*************************************************
    \begin{tikzpicture}
    %Axis
    \draw[thick,<->] (0,10) node[above]{Depreciación, Ahorro}--(0,0)--(10,0) node[right]{$k$}; 
    \node [below left] at (0,0) [$0$]; 
    %Line 
    \draw(0,0)--(9,9) node[right]{$(n + \delta) k$}; 
    \draw(0,0)--(8,10) node[right]{$(n' + \delta ) k$}; 
    
    \draw(6.1,0)--(6.1,6.1) node[right]{$A$};
    \draw(4.4,0)--(4.4,5.4) node[right]{$B$};
    %Curve 
    \draw(0,0) ..controls (1,5) and (5,6) .. (10,7) node[right]{$s*f(k)$};
    %\draw(0,0) ..controls (1,5) and (5,5) .. (10,6) node[right]{$sf(k)$};
    \end{tikzpicture}
    %*************************************************
    %_________________________________________________
    
    
    %\subpart Con el nivel de capital per – cápita de estado estacionario, la producción per – cápita, capital per – cápita de oro y el nivel de producción per – cápita de oro (de los incisos a) y b)); encontrar el consumo per – cápita de estado estacionario y de oro.

  \end{subparts}
\end{parts}

\begin{solution}
  
\end{solution}

%--------------------------------------------------------------------%
%--------------------------------------------------------------------%
%--------------------------------------------------------------------%
\miquestion
\begin{parts}
  \part El modelo AK viene dado por las siguientes ecuaciones:
  \[{Y_t} = A{K_t}\]
  \[{I_t} = {{\dot K}_t} + \delta {K_t}\]
  \[{S_t} = {I_t}\]
  
  \textbf{Stock de capital de Oro}
  
  $$k^{oro} = (\dfrac{(\delta +n)}{A * \alpha})^{\dfrac{1}{\alpha - 1}}$$
  
     $$\mathop k\limits^ \bullet   = s*f(k) - (n + a + \delta )*k$$
  
  \begin{subparts}
    \subpart
    Si el progreso tecnológico es 2, la tasa de ahorro es 20\%, la tasa de depreciación es del 18\% y la tasa demográfica es de 6\%. Encuentre la tasa de crecimiento del capital per – cápita y la tasa de crecimiento del producto per – cápita, realice una grafica.
    
    $$k^{oro} = (\dfrac{(0.18 +0.06)}{2  })$$
    $$k^{oro} = 0.12$$
    
    \subpart
    Suponga que el nivel de capital per – cápita es de 150, encuentre la producción per – cápita, realice una grafica
    
      \[{Y_t} = A{K_t}\]
      $$y_t = A*k_t$$
      $$y_t = 2*1.50$$
      $$y_t = 3$$
    
    \subpart
    Si el ahorro disminuye a un 15\%, ¿Qué sucede con la tasa de crecimiento del capital y del producto per – cápita? Elabore gráficos para respaldar su respuesta 
    
    \textbf{Tasa de crecimiento del Stock de Capital}
    
    
    
    $$\gamma _{\mathop k\limits^ \bullet}=0$$
    
    $$k_t=\dfrac{(n+\delta)}{A*s}$$
    
    $$k_t=\dfrac{(0.06+0.18)}{2*0.15}$$
    
    $$k_t= 0.8$$
    
    \subpart
    Si la tasa de depreciación cae a un 14\%, ¿Qué sucede con la tasa de crecimiento del capital y del producto per – cápita? Elabore gráficos para respaldar su respuesta
    
    $$k_t=\dfrac{(n+\delta)}{A*s}$$
    
     $$k_t=\dfrac{(0.06+0.14)}{2*0.2}$$
     
     $$k_t=0.5$$

  \end{subparts}
\end{parts}
%--------------------------------------------------------------------%
%--------------------------------------------------------------------%
%--------------------------------------------------------------------%
%\miquestion
%\begin{parts}
%    \part Considere el modelo de Solow del anterior ejercicio (I)
%        \begin{subparts}
%            \subpart    
%            Encuentre la velocidad de convergencia de este modelo. (Debe partir de la ecuación de acumulación de capital, luego aplicar la convergencia de Taylor):
            
%            \subpart
 %           Tome en cuenta que la elasticidad capital – producto per – cápita es 1/3, 0)  = 1, = 2, = 2\%, = 1\%  y = 3\%  (x  es  la  velocidad  de  crecimiento  de  la tecnología). Calcular cuántos años (t) tarda en llegar a la mitad.

 %       \end{subparts}
%\end{parts}
%--------------------------------------------------------------------%
%--------------------------------------------------------------------%
%--------------------------------------------------------------------%
\end{questions}
\end{document}
%--------------------------------------------------------------------%
%--------------------------------------------------------------------%
%--------------------------------------------------------------------%